\PassOptionsToPackage{margin=1in}{geometry}
\documentclass[11pt]{article}
\usepackage{lmodern}
\usepackage{wordlike}
\usepackage{setspace}
\usepackage{pdfpages}
\usepackage{grffile}
\usepackage{graphicx}
\usepackage{fancyhdr}
\usepackage{indentfirst}
\usepackage{wrapfig}
\pagestyle{fancy}
\lhead{CS 5714 UX Midterm Spring 2014}
\chead{}
\rhead{chris frisina}
\lfoot{Please redistribute as is or anonymized as you need}
\cfoot{}
\rfoot{LaTeX page: \thepage (not necesarily the submission page)}
\renewcommand{\headrulewidth}{0pt}
\renewcommand{\footrulewidth}{0pt}
% \doublespacing

\begin{document}

\title{CS 5714 UX Midterm Spring 2014}
\author{Chris Frisina}

\section*{Part 2} % 1 page
  \subsection{Question 1} 
  % (10 pts) Write a system concept statement for target systems above. This is a 100-to-150-word summary of the project, per the description of concept statement in the textbook and in our lectures. This is a high-level mission statement of a project—a synopsis or "boilerplate" description.  Include the name of the system, a description of the kinds of users expected, a brief statement of  what  users can do with it, and why it's useful (what problems  it  solves).  

  % What is the system name?  
  % –Who are the system users?  
  % –What will the system do? 
  % –What problem(s) will the system solve? (Be broad to include business objectives)
  % What is design vision and what are the emotional impact goals?
  % what experience will system provide to user
  % Audience broader than that of most other deliverables, including   
  % – Highlevel management 
  % – Marketing 
  % – Board of directors  
  % – Stockholders  
% – Even general public

  Graduate students at a large public educational institute have varying needs throughout their academic career.
  Their main contact is an overloaded faculty member who is struggling to accommodate all requests in a timely structured manner.

  I propose InkDrop, an automated proxy scheduler for assisting close community members in achieving meet-ups less than 5 minutes.
  While there are many communities that can benefit from InkDrop, it will initially support the academic community to accommodate rapidly changing multiplex organizational structures and  

  \subsection{Question 2}
  % (20pts) Make  an  initial Olow  model,  a “big  picture”  diagram of  the work  domain  and the entire  work  practice. Show  interconnections  among components  of  the work  domain. Show  work  Olow, information Olow, and all communications  among the components. Include non-human entities, such  as  a central database  and non-computer  communication Olow  such  as  via email,  telephone.

  
  \subsection{Question 3 - Work Roles}
  % . (20pts) Identify  all the work  roles in  the system  from  the interview transcript  above.  A work  role  is  deOined by  a corresponding job title or  a particular  type  of  work  assignment  or  a set of work  responsibilities. Work  roles don't always  involve using the system  being studied.  For each work  role, give  it  a name, and a brief description.
  Without making a complete assumption that Dr. MPQ @ Public Institution = Dr. Manuel 
  Roles marked with an asterick (\**) are explicitly mentioned in the interview.
  Roles without an asterick are inferred form the interview, however titles and duties have yet to be reified.
  \begin{description} % Numbered list example
  \item[Faculty Member \**]
  Dr. MPQ\\
  \item[Graduate Student Faculty Contact (GSFC) \**]
  Dr. MPQ\\
  This role is in charge of helping graduate students in several capacities including signing paperwork, holding brief meetings, and consultation.
  It is unclear if these roles are independent of a formal work role title despite potentially having different purposes.
  Many of the concerns aside from signatures are centered around information communication.

  \item[Additional Work Roles \**]
  Dr. MPQ\\
  It is mentioned that he has many other work roles, but they are not explicitly discussed or delineated.

  \item[Paperwork Retainer, Interested parties]
  Unknown\\
  The paperwork flow is not clearly defined.
  For example, does the student(s) continue handling the paperwork, or the signature the final step.
  This leaves open the questions of who retains the paperwork, who views the paperwork aside from the student and GSFC

  \item[Student (s)]
  There are graduate students who need paperwork signed, consultation regarding graduation, and other questions surrounding their education.
  
  \end{description}

  There are also other roles that must be considered, but aren't easily classified as a work role given the lack of inquiry, but closer to places and domains that have influence on this particular client's case:
  \begin{description} % Numbered list example
    \item[Higher Educational Public Institution \**]
    Not known\\
    \item[Physical Locations \**]
    Campus, and KWII\\
    \item[Smart Phones \**]


  \end{description}


  \subsection{Question 4}
  % (20pts) Make  a hierarchical  task  inventory diagram showing the task  structure for the product described in  the interview.  An  HTI Diagram captures  and catalogs  the hierarchical  relationships among the tasks and subtasks  that  must  be  supported in  the system. Remember  that  HTI does  NOT capture temporal  relationships.  

\section*{Part 3} % .5 page
  % (20pts) Make  a hierarchical  task  inventory diagram showing the task  structure for the product described in  the interview.  An  HTI Diagram captures  and catalogs  the hierarchical  relationships among the tasks and subtasks  that  must  be  supported in  the system. Remember  that  HTI does  NOT capture temporal  relationships.  


% \section*{Bibliography}
\newpage
\bibliographystyle{abbrv}
\bibliography{rmprop}

Omitting sections because of class assignment:

 - Research team / biographies \& abstracts of Co-PIs

 - Long Term Plans and Goals

\end{document}